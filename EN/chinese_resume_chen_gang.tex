
\documentclass{resume} % Use the custom resume.cls style

\usepackage[left=0.75in,top=0.6in,right=0.75in,bottom=0.6in]{geometry}
\usepackage{ctex}
\usepackage{fontspec}

\name{陈\hspace{0.4cm}钢} % Your name

%\address{北京海淀区中关村科学院南路6号}
\address{(+86)~18813188713 \\ cgangee@qq.com} % Your phone number and email

\begin{document}

\begin{rSection}{教育经历}
{\bf 中国科学院大学} \hfill {2015.9 - 2018.7} \\
计算机应用技术\ 硕士

{\bf 郑州大学} \hfill {2011.9 - 2015.7} \\
计算机科学与技术\ 学士
\end{rSection}


\begin{rSection}{项目经验}
\begin{rSubsection}{人车检测项目}{2016.11 - 现在}{}{}
\item 介绍:在室外监控场景的图片中检测人和车辆,并且预测检测到的人的属性和身体部件的位置,和车的品牌和种类。
\item 职责:负责人和车辆的检测模块,通过反卷积放大特征层和使用难例挖掘优化深度神经网络PVANET,使得在同样的每秒29帧的检测速度下,各个目标召回率得到提升,其中行人头部的召回率提升21\%。
\end{rSubsection}

\begin{rSubsection}{纹身识别项目}{2016.10 - 2017.5}{}{}
\item 介绍:在存有10万张图片的数据库中,检索不同角度、光照下拍摄的同一纹身图像。
\item 职责:清洗和标注纹身数据,并利用该数据训练纹身分类、检测和识别网络。利用纹身分类和检测网络去除不包含纹身的图片和区域,并用利用识别网络提取出纹身特征,进行纹身图像检索。
\end{rSubsection}


\begin{rSubsection}{校园图书分享APP}{2015.3 - 2015.5}{}{}
\item 介绍:可以发布、浏览和聊天的图书分享APP。
\item 职责:负责软件的总体设计和开发。
\end{rSubsection}
\end{rSection}


\begin{rSection}{荣誉奖励}
\begin{list}{$\cdot$}{\leftmargin=0em}
\itemsep -0.5em \vspace{0em}
\item 一等奖~~第5届蓝桥杯全国软件设计竞赛总决赛\hfill{2014}
\item 铜奖2~~次 第38届ACM/ICPC亚洲区域赛\hfill {2013}
\item 第一名3次~~第5 - 7届河南省大学生程序设计竞赛\hfill {2012 - 2014}
\item 国家励志奖学金 \hfill {2013}
%\item 优秀学生奖学金~~二等奖 \hfill {2014}
\end{list}
\end{rSection}


\begin{rSection}{实习经验}
\begin{rSubsection}{百度地图}{2015.3 - 2015.5}{研发实习生}{北京}
\item 从百度地图8千万的兴趣点中,优化了250万兴趣点的导航坐标,约3\%。
\item 利用导航GPS日志数据生成用户行车和导航轨迹,并用此来计算和分析偏航点。
\end{rSubsection}
\end{rSection}



\begin{rSection}{技能}
\begin{tabular}{ @{} >{\bfseries}l @{\hspace{6ex}} l }
编程语言 & C/C++, Python, Java, Shell\\
工具 & Caffe, Git, Vim\\
英语 & CET-4, CET-6 \\
\end{tabular}
\end{rSection}




\end{document}
